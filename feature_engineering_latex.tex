% Feature Engineering Section for Overleaf
% Copy this into your Overleaf document

\section{Feature Engineering Methodology}

\subsection{Motivation}

Traditional macroeconomic forecasting models often rely on a limited set of predictors, potentially missing valuable information contained in the broader economic landscape. To address this limitation, we implement a comprehensive feature engineering pipeline that transforms the raw FRED-MD dataset into an enriched feature space capturing temporal dynamics, cross-sectional relationships, and nonlinear patterns.

Our feature engineering approach is motivated by three key observations:
\begin{enumerate}
    \item \textbf{Temporal Dependencies}: Economic variables exhibit strong autocorrelation and momentum effects that can improve forecast accuracy.
    \item \textbf{Cross-Sectional Information}: Relationships between different macroeconomic indicators contain predictive power for inflation dynamics.
    \item \textbf{Nonlinear Patterns}: Economic relationships are often nonlinear, requiring transformations beyond simple lags.
\end{enumerate}

\subsection{Data Pipeline}

Our feature engineering pipeline consists of two main stages:

\subsubsection{Stage 1: Stationarity Transformations (FRED-MD)}

Following \citet{mccracken2016fred}, we apply appropriate transformations to ensure stationarity:
\begin{itemize}
    \item \textbf{Level} (Code 1): No transformation for stationary series
    \item \textbf{First Difference} (Code 2): $\Delta x_t = x_t - x_{t-1}$
    \item \textbf{Second Difference} (Code 3): $\Delta^2 x_t = \Delta x_t - \Delta x_{t-1}$
    \item \textbf{Log} (Code 4): $\log(x_t)$
    \item \textbf{Log Difference} (Code 5): $\Delta \log(x_t)$
    \item \textbf{Log Second Difference} (Code 6): $\Delta^2 \log(x_t)$
\end{itemize}

Starting with 126 raw macroeconomic variables from FRED-MD, these transformations yield 126 stationary predictors.

\subsubsection{Stage 2: Advanced Feature Engineering}

We apply a comprehensive set of feature engineering techniques to capture temporal and cross-sectional dynamics:

\paragraph{Rolling Statistics (4 windows: 3, 6, 12, 24 months)}
For each variable $x_t$ and window size $w$:
\begin{align}
    \text{Mean}_w(x_t) &= \frac{1}{w}\sum_{i=1}^{w} x_{t-i} \\
    \text{Std}_w(x_t) &= \sqrt{\frac{1}{w}\sum_{i=1}^{w} (x_{t-i} - \text{Mean}_w(x_t))^2} \\
    \text{Max}_w(x_t) &= \max_{i=1,\ldots,w} x_{t-i} \\
    \text{Min}_w(x_t) &= \min_{i=1,\ldots,w} x_{t-i}
\end{align}

\textbf{Data Leakage Prevention}: All rolling statistics use \texttt{.shift(1)} to exclude the current observation and \texttt{min\_periods=window} to ensure sufficient historical data.

\paragraph{Momentum Features (4 windows: 3, 6, 12, 24 months)}
\begin{equation}
    \text{Momentum}_w(x_t) = x_t - x_{t-w}
\end{equation}

\paragraph{Volatility Features (4 windows: 3, 6, 12, 24 months)}
\begin{equation}
    \text{Volatility}_w(x_t) = \text{Std}_w(x_t)
\end{equation}

\paragraph{Z-Score Normalization (1 window: 12 months)}
\begin{equation}
    \text{Z-Score}_{12}(x_t) = \frac{x_t - \text{Mean}_{12}(x_{t-1})}{\text{Std}_{12}(x_{t-1})}
\end{equation}

\paragraph{Cross-Sectional Correlation (1 window: 12 months)}
For each pair of variables $(x_t, y_t)$:
\begin{equation}
    \text{Corr}_{12}(x_t, y_t) = \frac{\text{Cov}_{12}(x_{t-1}, y_{t-1})}{\text{Std}_{12}(x_{t-1}) \cdot \text{Std}_{12}(y_{t-1})}
\end{equation}

\subsection{Feature Space Dimensions}

Table~\ref{tab:feature_counts} summarizes the feature space at each stage of our pipeline.

\begin{table}[htbp]
\centering
\caption{Feature Engineering Pipeline: Variable Counts}
\label{tab:feature_counts}
\begin{tabular}{lrr}
\toprule
\textbf{Stage} & \textbf{Features} & \textbf{Description} \\
\midrule
Raw FRED-MD & 126 & Original macroeconomic variables \\
Stationarity Transforms & 126 & After FRED-MD transformations \\
Rolling Statistics & 2,016 & 126 vars $\times$ 4 stats $\times$ 4 windows \\
Momentum & 504 & 126 vars $\times$ 4 windows \\
Volatility & 504 & 126 vars $\times$ 4 windows \\
Z-Scores & 126 & 126 vars $\times$ 1 window \\
Cross-Sectional Corr. & 1,260 & Selected pairs $\times$ 1 window \\
\midrule
\textbf{Total Engineered} & \textbf{4,410} & Before lagging \\
\bottomrule
\end{tabular}
\end{table}

\subsection{Smart Embedding Strategy}

\subsubsection{The Feature Explosion Problem}

A naive approach would lag all 4,410 engineered features by $h$ periods (where $h$ is the forecast horizon). This creates a \textit{feature explosion}:
\begin{equation}
    \text{Naive Features} = 4,410 \times (4 + h) \approx 20,000 \text{ features}
\end{equation}

This approach has critical flaws:
\begin{enumerate}
    \item \textbf{Redundancy}: Rolling statistics already incorporate historical information. A 12-month rolling mean at time $t$ uses data from $[t-12, t-1]$. Lagging it by 1 month gives $[t-13, t-2]$, which is 92\% redundant.
    \item \textbf{Computational Cost}: With 20,000 features, the correlation matrix requires $20,000^2 = 400$ million computations.
    \item \textbf{Runtime}: Feature selection and model training become prohibitively slow (80+ hours).
\end{enumerate}

\subsubsection{Our Solution: Smart Embedding}

We implement a \textit{smart embedding strategy} that separates raw feature lagging from feature engineering:

\begin{algorithm}[H]
\caption{Smart Embedding for Feature Engineering}
\begin{algorithmic}[1]
\State \textbf{Input:} Raw data $Y \in \mathbb{R}^{T \times 126}$, forecast horizon $h$
\State \textbf{Step 1:} Create lags of \textit{raw} features only
\State $X_{\text{raw}} \gets \text{embed}(Y, 4)$ \Comment{126 vars $\times$ 4 lags = 504 features}
\State \textbf{Step 2:} Engineer features on \textit{current} data (no lags)
\State $X_{\text{eng}} \gets \text{FeatureEngineer}(Y)$ \Comment{4,410 current features}
\State \textbf{Step 3:} Select appropriate lags for forecast horizon $h$
\If{$h = 1$}
    \State $X_{\text{lagged}} \gets X_{\text{raw}}$
\Else
    \State $X_{\text{lagged}} \gets X_{\text{raw}}[:, 126(h-1):126(h-1)+504]$
\EndIf
\State \textbf{Step 4:} Combine lagged raw + current engineered
\State $X \gets [X_{\text{lagged}} \mid X_{\text{eng}}]$ \Comment{504 + 4,410 = 4,914 features}
\State \textbf{Output:} Feature matrix $X \in \mathbb{R}^{T \times 4,914}$
\end{algorithmic}
\end{algorithm}

This approach reduces the feature count by 75\% (from $\sim$20,000 to $\sim$5,000) while preserving all relevant information.

\subsection{Feature Selection Pipeline}

Given the high-dimensional feature space, we implement a three-stage feature selection process:

\subsubsection{Stage 1: Constant Variance Removal}
Remove features with near-zero variance ($\sigma^2 < 10^{-6}$):
\begin{equation}
    \mathcal{F}_1 = \{j : \text{Var}(X_j) \geq 10^{-6}\}
\end{equation}

\subsubsection{Stage 2: Correlation-Based Filtering}
For highly correlated feature pairs ($|\rho| > 0.95$), retain the feature with higher variance:
\begin{equation}
    \text{If } |\text{Corr}(X_i, X_j)| > 0.95 \text{ and } \text{Var}(X_i) > \text{Var}(X_j), \text{ remove } X_j
\end{equation}

\subsubsection{Stage 3: Low Variance Removal}
Remove features in the bottom 5\% of variance distribution:
\begin{equation}
    \mathcal{F}_3 = \{j : \text{Var}(X_j) > Q_{0.05}(\text{Var}(X))\}
\end{equation}

\subsubsection{Hybrid Models: SelectKBest Pre-screening}

For our hybrid RF-FE model, we add an additional pre-screening step using statistical significance:
\begin{equation}
    F_j = \frac{\text{Var}(\mathbb{E}[y|X_j])}{\mathbb{E}[\text{Var}(y|X_j)]}
\end{equation}

We select the top 500 features with highest $F$-statistics before applying the three-stage selection, further improving computational efficiency.

\subsection{Computational Optimization}

\subsubsection{Parallelization Strategy}

We implement parallel processing using \texttt{joblib} with controlled parallelization:
\begin{itemize}
    \item \textbf{Random Forest \& XGBoost}: \texttt{n\_jobs=4} (prevents out-of-memory errors)
    \item \textbf{LSTM}: \texttt{n\_jobs=2} (GPU memory constraints)
    \item \textbf{Batch Processing}: Process forecasts in batches of 20 for memory management
\end{itemize}

\subsubsection{Memory Efficiency}

\begin{itemize}
    \item \textbf{Data Type Optimization}: Use \texttt{float32} instead of \texttt{float64} (50\% memory reduction)
    \item \textbf{Incremental Processing}: Feature engineering applied within rolling windows
    \item \textbf{Garbage Collection}: Explicit memory cleanup between iterations
\end{itemize}

\subsection{Performance Comparison}

Table~\ref{tab:performance} compares the naive and smart embedding approaches.

\begin{table}[htbp]
\centering
\caption{Performance Comparison: Naive vs. Smart Embedding}
\label{tab:performance}
\begin{tabular}{lrr}
\toprule
\textbf{Metric} & \textbf{Naive Approach} & \textbf{Smart Embedding} \\
\midrule
Total Features & $\sim$20,000 & $\sim$5,000 \\
Feature Reduction & -- & 75\% \\
Correlation Matrix Size & 400M elements & 25M elements \\
Memory Usage & $\sim$16 GB & $\sim$1 GB \\
Runtime (RF-FE) & 80 hours & 4--8 hours \\
Speedup & 1$\times$ & 10--20$\times$ \\
\bottomrule
\end{tabular}
\end{table}

\subsection{Data Leakage Prevention}

We implement strict temporal separation to prevent data leakage:

\begin{enumerate}
    \item \textbf{Rolling Statistics}: All calculations use \texttt{.shift(1)} to exclude current observation
    \item \textbf{Minimum Periods}: Set \texttt{min\_periods=window} to ensure sufficient historical data
    \item \textbf{No Backward Fill}: Use forward fill and zero-fill only; no \texttt{bfill()}
    \item \textbf{Expanding Window}: Training data strictly precedes forecast date
    \item \textbf{Feature Engineering Timing}: Applied separately within each rolling window
\end{enumerate}

\subsection{Implementation Details}

Our feature engineering pipeline is implemented in Python using:
\begin{itemize}
    \item \textbf{Core Libraries}: NumPy 1.24+, Pandas 2.0+
    \item \textbf{Machine Learning}: scikit-learn 1.3+
    \item \textbf{Deep Learning}: TensorFlow 2.13+ (for LSTM models)
    \item \textbf{Parallelization}: joblib 1.3+
    \item \textbf{Code Availability}: \url{https://github.com/NormidwitUpdatedII/48E-Shit}
\end{itemize}

All code is available in our GitHub repository with comprehensive documentation and reproducible examples.

\subsection{Summary}

Our feature engineering methodology transforms 126 raw macroeconomic variables into a rich feature space of approximately 5,000 predictors through:
\begin{itemize}
    \item Comprehensive temporal feature extraction (rolling statistics, momentum, volatility)
    \item Cross-sectional relationship modeling (correlations, z-scores)
    \item Smart embedding strategy to avoid feature explosion
    \item Rigorous three-stage feature selection
    \item Strict data leakage prevention
    \item Computational optimization (parallelization, memory efficiency)
\end{itemize}

This approach achieves a 10--20$\times$ speedup while maintaining forecast accuracy and ensuring temporal validity of all features.

\section{Experimental Design and Model Configurations}

\subsection{Data Samples}

We conduct our analysis using two distinct samples to ensure robustness:

\subsubsection{First Sample (2000--2025)}

\begin{itemize}
    \item \textbf{Total Observations}: 502 months (January 2000 -- October 2025)
    \item \textbf{Training Period}: 370 months (January 2000 -- October 2011)
    \item \textbf{Out-of-Sample Period}: 132 months (November 2011 -- October 2025)
    \item \textbf{Evaluation}: \texttt{nprev = 132}
    \item \textbf{Rationale}: Focuses on recent economic dynamics, including post-2008 financial crisis and COVID-19 pandemic
\end{itemize}

\subsubsection{Second Sample (1959--2025)}

\begin{itemize}
    \item \textbf{Total Observations}: 800 months (January 1959 -- October 2025)
    \item \textbf{Training Period}: 502 months (January 1959 -- October 2001)
    \item \textbf{Out-of-Sample Period}: 298 months (November 2001 -- October 2025)
    \item \textbf{Evaluation}: \texttt{nprev = 298}
    \item \textbf{Outlier Handling}: Includes dummy variable for COVID-19 period (March--April 2020)
    \item \textbf{Rationale}: Provides longer historical perspective spanning multiple business cycles
\end{itemize}

\subsection{Forecast Horizons}

We evaluate forecasts at 12 different horizons:
\begin{equation}
    h \in \{1, 2, 3, 4, 5, 6, 7, 8, 9, 10, 11, 12\} \text{ months}
\end{equation}

This allows us to assess both short-term (1--3 months) and medium-term (6--12 months) forecast performance.

\subsection{Target Variables}

We forecast two key inflation measures:
\begin{enumerate}
    \item \textbf{CPI}: Consumer Price Index for All Urban Consumers (CPIAUCSL)
    \item \textbf{PCE}: Personal Consumption Expenditures Price Index (PCEPI)
\end{enumerate}

Both variables are transformed to monthly growth rates following FRED-MD transformation codes.

\subsection{Model Configurations}

\subsubsection{Random Forest with Feature Engineering (RF-FE)}

\begin{table}[htbp]
\centering
\caption{Random Forest Hyperparameters}
\label{tab:rf_config}
\begin{tabular}{ll}
\toprule
\textbf{Parameter} & \textbf{Value} \\
\midrule
Number of Trees & 300 \\
Maximum Depth & 20 \\
Minimum Samples Split & 5 \\
Minimum Samples Leaf & 2 \\
Random State & 42 \\
Parallel Jobs & 4 \\
Bootstrap & True \\
\bottomrule
\end{tabular}
\end{table}

\textbf{Rationale}: These parameters balance model complexity with computational efficiency. The relatively deep trees (max depth = 20) allow capturing complex nonlinear relationships, while minimum sample constraints prevent overfitting.

\subsubsection{XGBoost with Feature Engineering (XGB-FE)}

\begin{table}[htbp]
\centering
\caption{XGBoost Hyperparameters}
\label{tab:xgb_config}
\begin{tabular}{ll}
\toprule
\textbf{Parameter} & \textbf{Value} \\
\midrule
Number of Estimators & 200 \\
Maximum Depth & 6 \\
Learning Rate & 0.05 \\
Subsample Ratio & 0.8 \\
Column Sample by Tree & 0.8 \\
Random State & 42 \\
Parallel Jobs & 4 \\
Verbosity & 0 \\
\bottomrule
\end{tabular}
\end{table}

\textbf{Rationale}: Conservative learning rate (0.05) with moderate tree depth (6) and subsampling (0.8) to prevent overfitting. The 200 estimators provide sufficient boosting iterations while maintaining computational efficiency.

\subsubsection{LSTM with Feature Engineering (LSTM-FE)}

\begin{table}[htbp]
\centering
\caption{LSTM Architecture and Hyperparameters}
\label{tab:lstm_config}
\begin{tabular}{ll}
\toprule
\textbf{Parameter} & \textbf{Value} \\
\midrule
\multicolumn{2}{l}{\textit{Architecture}} \\
LSTM Layer 1 Units & 64 \\
LSTM Layer 1 Activation & tanh \\
LSTM Layer 1 Return Sequences & True \\
Dropout Rate (Layer 1) & 0.2 \\
LSTM Layer 2 Units & 32 \\
LSTM Layer 2 Activation & tanh \\
Dense Layer Units & 16 \\
Dense Layer Activation & ReLU \\
Output Layer & 1 (linear) \\
\midrule
\multicolumn{2}{l}{\textit{Training}} \\
Optimizer & Adam \\
Learning Rate & 0.001 \\
Loss Function & MSE \\
Epochs & 100 \\
Batch Size & 16 \\
Early Stopping Patience & 10 \\
Parallel Jobs & 2 \\
\midrule
\multicolumn{2}{l}{\textit{Feature Selection}} \\
Maximum Features & 200 \\
Selection Method & Variance-based \\
Lookback Window & 4 \\
\bottomrule
\end{tabular}
\end{table}

\textbf{Rationale}: Two-layer LSTM architecture with decreasing units (64 → 32) captures temporal dependencies at multiple scales. Dropout (0.2) and early stopping prevent overfitting. Feature limit (200) ensures computational feasibility while retaining most informative predictors.

\subsubsection{Hybrid RF-FE Model}

Our hybrid model combines the smart embedding strategy with additional optimizations:

\begin{table}[htbp]
\centering
\caption{Hybrid RF-FE Configuration}
\label{tab:hybrid_config}
\begin{tabular}{ll}
\toprule
\textbf{Component} & \textbf{Configuration} \\
\midrule
\multicolumn{2}{l}{\textit{Feature Engineering}} \\
Smart Embedding & Yes (504 lagged + 4,410 current) \\
Data Type & float32 \\
\midrule
\multicolumn{2}{l}{\textit{Feature Selection}} \\
SelectKBest Pre-screening & 500 features \\
Scoring Function & F-regression \\
3-Stage Selection & Yes \\
Final Features & $\sim$400--500 \\
\midrule
\multicolumn{2}{l}{\textit{Random Forest}} \\
Number of Trees & 200 \\
Maximum Depth & 12 \\
Random State & 42 \\
Parallel Jobs & 4 \\
\midrule
\multicolumn{2}{l}{\textit{Benchmarking}} \\
Random Walk Baseline & Yes \\
Relative RMSE Reporting & Yes \\
\bottomrule
\end{tabular}
\end{table}

\textbf{Key Innovation}: The hybrid model adds SelectKBest pre-screening before the 3-stage selection, reducing computational cost while maintaining predictive power. Random Walk benchmarking provides interpretable performance metrics.

\subsection{Feature Selection Thresholds}

\begin{table}[htbp]
\centering
\caption{Feature Selection Threshold Values}
\label{tab:selection_thresholds}
\begin{tabular}{llr}
\toprule
\textbf{Stage} & \textbf{Threshold Type} & \textbf{Value} \\
\midrule
Stage 1 & Constant Variance & $10^{-6}$ \\
Stage 2 & Correlation & 0.95 \\
Stage 3 & Low Variance Percentile & 5\% \\
Hybrid & SelectKBest $k$ & 500 \\
LSTM & Maximum Features & 200 \\
\bottomrule
\end{tabular}
\end{table}

\subsection{Rolling Window Evaluation}

We employ an expanding window approach for out-of-sample evaluation:

\begin{algorithm}[H]
\caption{Rolling Window Forecast Evaluation}
\begin{algorithmic}[1]
\State \textbf{Input:} Data $Y$, training size $n_{\text{train}}$, test size $n_{\text{test}}$, horizon $h$
\For{$i = 1$ to $n_{\text{test}} - h$}
    \State $Y_{\text{train}} \gets Y[1 : n_{\text{train}} + i]$ \Comment{Expanding window}
    \State Apply feature engineering to $Y_{\text{train}}$
    \State Train model on engineered features
    \State $\hat{y}_{n_{\text{train}}+i+h} \gets$ Forecast for horizon $h$
    \State $y_{n_{\text{train}}+i+h} \gets$ Actual value
    \State Store forecast and actual
\EndFor
\State Calculate RMSE, MAE, and other metrics
\State \textbf{Output:} Forecast errors and performance metrics
\end{algorithmic}
\end{algorithm}

\textbf{Key Features}:
\begin{itemize}
    \item \textbf{Expanding Window}: Training set grows with each iteration, using all available historical data
    \item \textbf{No Look-Ahead Bias}: Feature engineering applied separately within each window
    \item \textbf{Realistic Evaluation}: Mimics real-time forecasting scenario
\end{itemize}

\subsection{Computational Environment}

\begin{table}[htbp]
\centering
\caption{Computational Specifications}
\label{tab:compute_specs}
\begin{tabular}{ll}
\toprule
\textbf{Component} & \textbf{Specification} \\
\midrule
\multicolumn{2}{l}{\textit{Software}} \\
Python Version & 3.11+ \\
NumPy & 1.24+ \\
Pandas & 2.0+ \\
scikit-learn & 1.3+ \\
XGBoost & 2.0+ \\
TensorFlow & 2.13+ \\
joblib & 1.3+ \\
\midrule
\multicolumn{2}{l}{\textit{Parallelization}} \\
RF/XGB Workers & 4 \\
LSTM Workers & 2 \\
Batch Size & 20 forecasts \\
\midrule
\multicolumn{2}{l}{\textit{Memory Management}} \\
Data Type & float32 \\
Garbage Collection & Explicit \\
Expected Memory & 1--2 GB \\
\bottomrule
\end{tabular}
\end{table}

\subsection{Performance Metrics}

We evaluate forecast performance using multiple metrics:

\subsubsection{Root Mean Squared Error (RMSE)}
\begin{equation}
    \text{RMSE} = \sqrt{\frac{1}{n}\sum_{t=1}^{n}(y_t - \hat{y}_t)^2}
\end{equation}

\subsubsection{Mean Absolute Error (MAE)}
\begin{equation}
    \text{MAE} = \frac{1}{n}\sum_{t=1}^{n}|y_t - \hat{y}_t|
\end{equation}

\subsubsection{Relative RMSE (Hybrid Model)}
\begin{equation}
    \text{Relative RMSE} = \frac{\text{RMSE}_{\text{Model}}}{\text{RMSE}_{\text{Random Walk}}}
\end{equation}

Values less than 1.0 indicate the model outperforms the naive Random Walk benchmark.

\subsection{Reproducibility}

To ensure reproducibility:
\begin{itemize}
    \item \textbf{Random Seeds}: Fixed at 42 for all models
    \item \textbf{Data Version}: FRED-MD vintage November 2025
    \item \textbf{Code Repository}: \url{https://github.com/NormidwitUpdatedII/48E-Shit}
    \item \textbf{Documentation}: Comprehensive README and inline comments
    \item \textbf{Version Control}: All changes tracked via Git
\end{itemize}

\subsection{Runtime Performance}

Table~\ref{tab:runtime} summarizes the computational efficiency of our optimized pipeline.

\begin{table}[htbp]
\centering
\caption{Runtime Performance (First Sample, All Horizons)}
\label{tab:runtime}
\begin{tabular}{lrrr}
\toprule
\textbf{Model} & \textbf{Before} & \textbf{After} & \textbf{Speedup} \\
\midrule
RF-FE & 80 hours & 4--8 hours & 10--20$\times$ \\
XGB-FE & 80 hours & 4--8 hours & 10--20$\times$ \\
LSTM-FE & 100+ hours & 8--12 hours & 10--15$\times$ \\
Hybrid RF-FE & -- & 4--8 hours & -- \\
\bottomrule
\end{tabular}
\end{table}

\subsection{Summary of Configurations}

Our experimental design features:
\begin{itemize}
    \item Two distinct samples (short-term: 2000--2025; long-term: 1959--2025)
    \item 12 forecast horizons (1--12 months)
    \item 2 target variables (CPI, PCE)
    \item Carefully tuned hyperparameters for each model
    \item Rigorous expanding window evaluation
    \item Comprehensive performance metrics including Random Walk benchmarking
    \item Full reproducibility with version-controlled code
\end{itemize}

This configuration ensures robust evaluation across different time periods, forecast horizons, and model specifications.
